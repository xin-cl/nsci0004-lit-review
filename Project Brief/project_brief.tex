\documentclass{article}
\usepackage[top=25mm, bottom=25mm, right=25mm, left=25mm, paper=a4paper]{geometry}
\usepackage{parskip}

\title{Project Brief}
\author{Xin Cheng Liu}
\date{}

\begin{document}
\maketitle

\section{Project Title}
Using nuclear tracks in solids to detect cosmic rays.

\section{Project Description}
Cosmic rays are high energy particles originating from supernovae or other astrophysical events and are constantly
coming into contact with the Earth. 87\% of primary cosmic rays are protons or hydrogen nuclei; 10\% are helium 
nuclei and the 1\% are heavier nuclei\cite{dunai2010cosmogenic}. When primary cosmic rays interact with the Earth's
atmosphere, they produce secondary cosmic rays that cascade down to produce three main components of secondary 
particles: electromagnetic, mesonic and nucleonic. 

When these high energy particles interact with certain minerals/solids, they leave behind latent damage known as
nuclear tracks. These tracks are microscopic and to reveal them, chemical or plasma etching is used to widen the 
tracks. 

This process has been used on many extraterrestrial objects such as meteorites and helemts from Apollo missions, 
as they have a higher potential of coming in contact with the heavier primary cosmic rays, compared to terrestrial 
minerals. Such research can be found in \cite{fleischer1967tracks}, \cite{aleksandrov2011meteorites}, \cite{bull1976cosmic}
and \cite{comstock1971cosmic}.

However, this literature review is interested in taking this process and carrying it out terrestrially. Through
placement of suitable solids for detecting cosmic rays at high altitudes, such as on mountains. We can probe the 
change in cosmic ray flux through nuclear track density at different altitudes. 

Similar work has been carried out by \cite{basu2015observation}, where they placed Solid State Nuclear Track
Detectors (SSNTDs) at Darjeeling, India. 

\bibliographystyle{plain}
\bibliography{biblio}
\end{document}