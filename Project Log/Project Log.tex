\documentclass[a4paper]{article}
\usepackage[utf8]{inputenc}
\usepackage[left=1in,right=1in,top=1in,bottom=1in]{geometry}
\usepackage{siunitx, amsmath,
            booktabs, tabularray,
            pdflscape, parskip, graphicx, hyperref, float}
\UseTblrLibrary{booktabs, siunitx}

\title{Reading Log}
\author{Xin Cheng Liu}
\date{}

\begin{document}

\maketitle

\section*{Week 1 - 12/11/2025}
\href{https://journals.aps.org/prd/abstract/10.1103/PhysRevD.101.103017}
{Paleodetectors for Galactic supernova neutrinos}, 2020 - Cited by 36

\href{https://iopscience.iop.org/article/10.3847/1538-4357/aa6c57}
{A Supernova at 50 pc: Effects on the Earth's Atmosphere and Biota}, 2017 - Cited by 75

\href{https://royalsocietypublishing.org/doi/epdf/10.1098/rsta.2019.0562}
{The lunar surface as a recorder of astrophysical processes}, 2021 - Cited by 32
\begin{itemize}
    \item Review on using the lunar surface as a recorder for solar and galactic cosmic rays
    \item Preservation of records: \\
    - eruption of low-viscosity basaltic lava flow \\
    - deposition of pyroclastic deposits around sites of explosive volcanism \\
    - emplacement of impact crater ejecta blankets
    \item Locating and accessing records in subsurface layers
\end{itemize}

\section*{Week 2 - 19/11/2025}
\href{https://www.lpi.usra.edu/publications/books/lunar_sourcebook/pdf/LunarSourceBook.pdf}
{Lunar sourcebook: A user's guide to the Moon}, 1991 - Cited by 2610
\begin{itemize}
    \item Chapter 3: Galactic Cosmic Rays \\
        - GCR particles with energies below ~1015 eV come from our galaxy, and their flux at the Earth is very 
        isotropic (Simpson, 1983)
    \item Chapter 5: Lunar Minerals
\end{itemize}

\href{run:./Cosmic-ray-produced noble gases in meteorites.pdf}
{Cosmic-ray-produced noble gases in meteorites}, 2002 - Cited by 242
\begin{itemize}
    \item  THE COSMIC RAY FLUX IN TIME page 159
\end{itemize}

\href{https://www.sciencedirect.com/science/article/pii/S1350448701001123?casa_token=n9SR6JRiPREAAAAA:3Y5BDISDCK-erSPYhpfhXMRHQE8ly-JkevWwXQsPxOWs_MrL7UI_pV9_RxZP9YMzL-Ugeh1EADY#aep-section-id8}
{Nuclear tracks: A success story of the 20th century}, 2001 - Cited by 63
\begin{itemize}
    \item Review on how nuclear tracks have been used in various fields and potential future studies
\end{itemize}

\href{https://www.science.org/doi/epdf/10.1126/science.219.4581.127}
{Cosmic-Ray Record in Solar System Matter}, 1983 - Cited by 262
\begin{itemize}
    \item The heliosphere is one of three screens, together with the Earth’s magnetic field and the Earth’s
    atmosphere, that modulate the cosmic ray flux at the surface of the Earth.
    \item The loss of heliospheric modulation would lead to flux increases of 10–100× at energies of 10–100 MeV at 
    the top of the terrestrial magnetosphere
\end{itemize}

\href{https://arxiv.org/abs/astro-ph/0601117}
{The Local Interstellar Medium}, 2006 - Cited by 33
\begin{itemize}
    \item If we look at a very nearby star, in the direction of the historical solar trajectory (Dehnen \& Binney 
    1998), the observed LISM absorption should provide information on the nature of the LISM that the Sun 
    encountered only a short time ago.
    \item The Sun’s ISM history could then be converted into a cosmic ray flux history, based on the heliospheric
    response to the historical interstellar density profile.
\end{itemize}

\href{https://www.nature.com/articles/217051a0}
{Duration of Sensitive Period for Track Recording in Mica}, 1968 - Cited by 24

\href{https://www.sciencedirect.com/science/article/abs/pii/0370269367900214?fr=RR-2&ref=pdf_download&rr=9a50bd84692e63cd}
{The observation in mica of tracks of charged particles from neutrino interactions}, 1967 - Cited by 30

\href{https://www.sciencedirect.com/science/article/pii/1359018988900982?fr=RR-2&ref=pdf_download&rr=9a50bd8bdf6f63cd}
{Identification and selection criteria for charged lepton tracks in mica}, 1988 - Cited by 48

\section*{Week 3 - 26/11/2025}
\href{https://www.sciencedirect.com/science/article/abs/pii/0370269367900214?fr=RR-2&ref=pdf_download&rr=9a50bd84692e63cd}
{The observation in mica of tracks of charged particles from neutrino interactions}, 1967 - Cited by 30
\begin{itemize}
    \item Mica exhibit internal decoration of oxides of iron. Random spots and sets of straight lines in 001.
    \item Majority of lines orientated in directions related directly to the crystal internal structure.
    \item Small minority exist which are orientated essentially at random.
    \item Suggests the minority is caused by passage of charged high energy particles through the crystal during a 
    particular stage in the history of the crystal.
\end{itemize}

\href{https://www.sciencedirect.com/science/article/pii/037026936890600X?ref=pdf_download&fr=RR-2&rr=9a8353e29c80bee2}
{Ancient Cosmic Ray Tracks in mica?}, 1968
\begin{itemize}
    \item Comments on "The observation in mica of tracks of charged particles from neutrino interactions"
    \item Expecting: \\
    - Their distribution should be consistent with that expected for neutrino-induced muons, i.e. nearly isotropic 
    at the energies in question or, alternatively, if the depth were such that the charged particles could be 
    atmospheric muons, then the distribution should be consistent with the known sharp angular distribution for this
    component. \\
    - Their lack of straightness should be consistent with the effect of Coulomb scattering; thus, the distribution 
    in scattering angle for successive segments of track should be near Gaussian with only a very small single scattering
    'tail'. 
    \item Minority lines don't appear to be random. Two preferred directions between 3 and 4\unit{\degree} and the other between 
    6 and 7\unit{\degree}.
    \item Orientation measurements supports the first requirement for genuine tracks.
    \item Scattering is not due to Coulomb scattering but could be Rutherford scattering.
    \item Not been able to substantiate the suggestion by Russell that the minority lines are tracks left by 
    neutrino-induced muons or by any other kind of charged particles. 

\end{itemize}

\section*{Week 4 - 03/12/2025}

\section*{Week 5 - 10/12/2025}

\section*{Week 6 - 17/12/2025}

\end{document}